% search for "LANG" in this file in order to see how to use a
% different language

\pdfcompresslevel=9
\NeedsTeXFormat{LaTeX2e}
\documentclass[10pt,a4paper]{article}

\usepackage[margin=0.8cm,noheadfoot,a4paper]{geometry}
\usepackage{calc}
\usepackage[english,ngerman,spanish]{babel} % language support
\usepackage[latin1]{inputenc}
\usepackage[T1]{fontenc}
\usepackage{ae}
\usepackage{bembo}
\usepackage{ragged2e}
\usepackage{paralist}
%\usepackage{eso-pic}
\usepackage{graphicx}
\usepackage{rcsinfo}

\usepackage{hyperref}
\hypersetup {
    pdfpagemode=Full,
    colorlinks,
    urlcolor=black,
    pdftitle={GPE Flyer},
    pdfauthor={Colin Marquardt},
    pdfsubject={$Revision$, $Date$},
    pdfkeywords={GPE Palmtop Environment marketing features technology flyer},
}

% ------------------------------------------------------------------------------
% accompanying wiki page: http://handhelds.org/z/wiki/GPEMarketing

\newcommand{\flyertext}{%
\rcsInfo $Id$
%\AddToShipoutPicture{%
%\includegraphics[width=\textwidth/2-1cm]{/opt/applications/gpe_cvs_dev/gpe/gpe-logo/gpe-logo-simple.jpg}}
\textbf{\large
  \includegraphics[height=\lineheight*3/5]{/opt/applications/gpe_cvs_dev/gpe/gpe-logo/gpe-logo-simple.jpg}
GPE: The GPE Palmtop Environment
  \includegraphics[height=\lineheight*3/5]{/opt/applications/gpe_cvs_dev/gpe/gpe-logo/gpe-logo-simple.jpg}
}

\smallskip
\small{
  \href{http://gpe.handhelds.org}{\texttt{http://gpe.handhelds.org}} --
  \href{mailto:gpe@handhelds.org}{\texttt{mailto:gpe@handhelds.org}} -- 
  \href{irc://freenode.net:ircd/\#gpe}{\texttt{irc://freenode.net:ircd/\#gpe}}
}

\smallskip
A Free Software GUI environment targeted at palmtop/handheld
computers, such as the HP iPAQ and the Sharp Zaurus. Think �GNOME
for handhelds�.

% GPE provides the infrastructure for developing palmtop applications,
% and includes core Personal Information Management (PIM)
% applications.

\smallskip
\textbf{Key Features}
\begin{compactitem}
   \item Custom GTK-based widget library (LGPL)
   \item Applications: calendar, to do list, address book, media
     player, notebook, games, and more (GPL)
   \item System utilities including Wi-Fi and Bluetooth manager
   \item Internationalization
   \item Display migration: �teleport� running applications to other
     displays
\end{compactitem}

\smallskip
\textbf{Technology}
\begin{compactitem}
   \item X Window System{\texttrademark}{} server with Render and
     Resize/Rotate extension (e.\,g. freedesktop.org X server, kdrive)
   \item fontconfig
   \item XFT2 with subpixel anti-aliased font support
   \item GTK+~2.2, featuring bidirectional text, display migration
     etc.
   \item freedesktop.org standards support
   \item Matchbox, a lightweight window manager designed for
     embedded devices
   \item SQLite for data storage
   \item GStreamer multimedia framework
   \item Packages available for Familiar Linux. OpenEmbedded support
     upcoming.
   \item \textbf{NEW! Synchronisation for Contacts, Todo and
     Calendar} in both directions with multisync.
\end{compactitem}

\smallskip
\textit{Users and developers welcome!}%
\hfill\tiny{\rcsInfoRevision, \rcsInfoYear-\rcsInfoMonth-\rcsInfoDay}
}

% ------------------------------------------------------------------------------

\begin{document}

\thispagestyle{empty}
\pagestyle{empty}
\RaggedRight
\begin{table}
  \begin{tabular}{p{\textwidth/2-1cm}p{1cm}p{\textwidth/2-1cm}}%
    \flyertext{} & & \flyertext{}\\[1cm]
    \flyertext{} & & \flyertext{}
%    \\\vspace{1cm}%
  \end{tabular}
\end{table}
%% use something like this for flyers that are 3-up on landscape paper:
% \begin{table}
%   \begin{tabular}{p{\textwidth/2-1cm}p{1cm}p{\textwidth/2-1cm}p{1cm}p{\textwidth/2-1cm}}%
%     %% input another language file (that is using \renewcommand{\flyertext})
%     % \input{flyertext_es}
%     \flyertext{} & & \flyertext{} & & \flyertext{}  
%   \end{tabular}
% \end{table}

\clearpage
%% input another language file (that is using \renewcommand{\flyertext})
% This German version from Colin Marquardt is translated against
% gpe-flyer.tex Revision: 1.8

\renewcommand{\flyertext}{%
\selectlanguage{ngerman}
%
%\AddToShipoutPicture{%
%\includegraphics[width=\textwidth/2-1cm]{/opt/applications/gpe_cvs_dev/gpe/gpe-logo/gpe-logo-simple.jpg}}
\textbf{\large
  \includegraphics[height=\lineheight*3/5]{/opt/applications/gpe_cvs_dev/gpe/gpe-logo/gpe-logo-simple.jpg}
GPE: The GPE Palmtop Environment
  \includegraphics[height=\lineheight*3/5]{/opt/applications/gpe_cvs_dev/gpe/gpe-logo/gpe-logo-simple.jpg}
}

\smallskip
\small{
  \href{http://gpe.handhelds.org}{\texttt{http://gpe.handhelds.org}} --
  \href{mailto:gpe@handhelds.org}{\texttt{mailto:gpe@handhelds.org}} -- 
  \href{irc://freenode.net:ircd/\#gpe}{\texttt{irc://freenode.net:ircd/\#gpe}}
}

\smallskip

Eine GUI-Umgebung, die auf Palmtops und Handheld-Computer wie den HP
iPAQ und den Sharp Zaurus abzielt. �GNOME f�r Handhelds� ist nicht
weit weg von der Wahrheit. GPE ist Freie Software.

\smallskip
\textbf{Eigenschaften}
\begin{compactitem}
   \item Eigene GTK-basierte Widget Library (LGPL)
   \item Applikationen: Kalendar, To-do-Liste, Adre�buch,
     Medienspieler, Notizbuch, Spiele und mehr (GPL)
   \item Systemwerkzeuge wie Wi-Fi- und Bluetooth-Verwaltung
   \item Internationalisierung
   \item �Display migration�: laufende Anwendungen auf andere
  Displays verschieben
\end{compactitem}

\smallskip
\textbf{Technik}
\begin{compactitem}
   \item X Window System{\texttrademark}{}-Server mit Render- und
     Resize/Rotate-Erweiterungen (z.\,B. freedesktop.org-X-Server, kdrive)
   \item fontconfig
   \item XFT2 mit Subpixel-anti-aliasing f�r Schriften
   \item GTK+~2.2, bidirektionaler Text, �display migration�
     usw.
   \item Unterst�tzung von freedesktop.org-Standards
   \item Matchbox, ein schlanker Fenstermanager, der f�r
     Embedded-Ger�te entworfen wurde
   \item SQLite zur Datenspeicherung
   \item GStreamer Multimedia-Framework
   \item Pakete f�r Familiar Linux verf�gbar;
     OpenEmbedded-Unterst�tzung in Arbeit.
   \item \textbf{NEU! Synchronisierung f�r Kontakte, To-Do und
     Kalender} in beide Richtungen mittels multisync.
\end{compactitem}

\smallskip
\textit{Anwender und Entwickler willkommen!}%
\hfill\tiny{\rcsInfoRevision, \rcsInfoYear-\rcsInfoMonth-\rcsInfoDay}
}

%%% Local Variables: 
%%% mode: latex
%%% TeX-master: "gpe-flyer"
%%% End: 


\RaggedRight
\begin{table}
  \begin{tabular}{p{\textwidth/2-1cm}p{1cm}p{\textwidth/2-1cm}}%
    \flyertext{} & & \flyertext{}\\[1cm]
    \flyertext{} & & \flyertext{}
%    \\\vspace{1cm}%
  \end{tabular}
\end{table}

\clearpage
%% input another language file (that is using \renewcommand{\flyertext})
% this Spanish version from J.Manrique Lopez de la Fuente
% is translated against gpe-flyer.tex Revision: 1.4, Date: 2003/03/05 18:33:14

\renewcommand{\flyertext}{%
%% I don't have spanish hyphenation patterns installed [CM]
%\selectlanguage{spanish}
%
%\AddToShipoutPicture{%
%\includegraphics[width=\textwidth/2-1cm]{/opt/applications/gpe_cvs_dev/gpe/gpe-logo/gpe-logo-simple.jpg}}
\textbf{\large
  \includegraphics[height=\lineheight*3/5]{/opt/applications/gpe_cvs_dev/gpe/gpe-logo/gpe-logo-simple.jpg}
GPE: The GPE Palmtop Environment
  \includegraphics[height=\lineheight*3/5]{/opt/applications/gpe_cvs_dev/gpe/gpe-logo/gpe-logo-simple.jpg}
}

\smallskip
\small{
  \href{http://gpe.handhelds.org}{\texttt{http://gpe.handhelds.org}} --
  \href{mailto:gpe@handhelds.org}{\texttt{mailto:gpe@handhelds.org}} -- 
  \href{irc://freenode.net:ircd/\#gpe}{\texttt{irc://freenode.net:ircd/\#gpe}}
}

\smallskip
Interface gr{\'a}fico libre orientado a ordenadores de bolsillo,
como los HP iPAQ y Sharp Zaurus. Piensa en {\guillemotright}GNOME para ordenadores
de bolsillo{\guillemotleft}.

% GPE provides the infrastructure for developing palmtop applications,
% and includes core Personal Information Management (PIM)
% applications.

\smallskip
\textbf{Caracter{\'\i}sticas Clave}
\begin{compactitem}
   \item Librer{\'\i}a de componentes basada en GTK (LGPL)
   \item Aplicaciones: calendario, tareas, libro de direccioens, reproductor
     multimedia, notas, juegos, y m{\'a}s (GPL)
   \item Utilidades del sistema como administraci{\'o}n de Wi-Fi y Bluetooth
   \item Internacionalizaci{\'o}n
   \item Migraci{\'o}n de pantalla: {\guillemotright}teleport{\guillemotleft} ejecutando aplicaciones a otras
     pantallas
\end{compactitem}

\smallskip
\textbf{Tecnolog{\'\i}a}
\begin{compactitem}
   \item Sistema X Window {\texttrademark}{} (freedesktop.org X
     server, kdrive) con las extensiones Render y
     Redimensionar/Rotar
   \item fontconfig
   \item XFT2 con soporte de fuentes subpixel anti-aliased
   \item GTK+~2.2, incluyendo soporte para texto bidireccional, migraci{\'o}n de pantalla,
     etc.
   \item Soporte de los est{\'a}ndares de freedesktop.org
   \item Matchbox, ligero gestor de ventanas dise{\~n}ado para dispositivos
     empotrados
   \item SQLite para el archivo de datos
   \item GStreamer multimedia framework
   \item Paquetes disponibles para Familiar Linux. Soporte para OpenEmbedded
     en camino.
\end{compactitem}

\smallskip
\textit{{\textexclamdown}Usarios y desarrolladores son bienvenidos!}%
\hfill\tiny{\rcsInfoRevision, \rcsInfoYear-\rcsInfoMonth-\rcsInfoDay}
}

%%% Local Variables: 
%%% mode: latex
%%% TeX-master: "gpe-flyer"
%%% End: 


\RaggedRight
\begin{table}
  \begin{tabular}{p{\textwidth/2-1cm}p{1cm}p{\textwidth/2-1cm}}%
    \flyertext{} & & \flyertext{}\\[1cm]
    \flyertext{} & & \flyertext{}
%    \\\vspace{1cm}%
  \end{tabular}
\end{table}


\end{document}



% From: Andreas Matthias <amat@KABSI.AT>
% Subject: Re: Subject: 4 gleiche Flyer auf ein A4-Blatt?
% To: TEX-D-L@LISTSERV.DFN.DE
% User-Agent: Gnus/5.09 (Gnus v5.9.0) Emacs/21.3.50
% X-Gnus-Mail-Source: file:/var/spool/mail/colin
% Message-ID:  <m3u1ekkpjz.fsf@h081217002064.dyn.cm.kabsi.at>
% 
% Colin Marquardt wrote:
% 
% > Herbert Voss <Herbert.Voss@ALUMNI.TU-BERLIN.DE> writes:
% > 
% >> Colin Marquardt schrieb:
% >> 
% >>> Ich muss einen Flyer setzen
% >>> (<http://handhelds.org/z/wiki/GPEMarketing>, wen es interessiert).
% >> 
% >> z.B. in eine Tabelle setzen
% > 
% > So habe ich das letztendlich auch gemacht.
% 
% 
% Ich hatte eigentlich gehofft, dass sich irgendwo noch ein
% Tool f�r diese Aufgabe findet, weshalb ich meine primitive
% Bastlerl�sung nicht posten wollte. Sie ist aber vielleicht
% doch etwas einfacher und allgemeiner als alles mit Tabellen
% zu versuchen. Daher hier doch kurz ein Beispiel, wie man es
% mit pdfpages.sty in 3 Schritten zusammenpfuschen kann.
% 
% Ciao
% Andreas
% 
% 
% %%% %%% dummy.tex %%% %%%
% \documentclass{article}
% \usepackage[a6paper]{geometry}
% \begin{document}
% First page \ldots \vfill First page \ldots\newpage
% Second page \ldots \vfill Second page \ldots
% \end{document}
% %%% %%% dummy.tex %%% %%%
% 
% 
% %%% %%% dummy_2.tex %%% %%%
% \documentclass[a4paper]{article}
% \usepackage{pdfpages}
% \begin{document}
% \includepdf[pages=-, doublepages, nup=1x2, landscape]{dummy.pdf}
% \end{document}
% %%% %%% dummy_2.tex %%% %%%
% 
% 
% %%% %%% dummy_4.tex %%% %%%
% \documentclass[a4paper]{article}
% \usepackage{pdfpages}
% \begin{document}
% \includepdf[pages=-, doublepages, nup=1x2]{dummy_2.pdf}
% \end{document}
% %%% %%% dummy_4.tex %%% %%%

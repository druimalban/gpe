% This German version from Colin Marquardt is translated against
% gpe-flyer.tex Revision: 1.8

\renewcommand{\flyertext}{%
\selectlanguage{ngerman}
%
%\AddToShipoutPicture{%
%\includegraphics[width=\textwidth/2-1cm]{/opt/applications/gpe_cvs_dev/gpe/gpe-logo/gpe-logo-simple.jpg}}
\textbf{\large
  \includegraphics[height=\lineheight*3/5]{/opt/applications/gpe_cvs_dev/gpe/gpe-logo/gpe-logo-simple.jpg}
GPE: The GPE Palmtop Environment
  \includegraphics[height=\lineheight*3/5]{/opt/applications/gpe_cvs_dev/gpe/gpe-logo/gpe-logo-simple.jpg}
}

\smallskip
\small{
  \href{http://gpe.handhelds.org}{\texttt{http://gpe.handhelds.org}} --
  \href{mailto:gpe@handhelds.org}{\texttt{mailto:gpe@handhelds.org}} -- 
  \href{irc://freenode.net:ircd/\#gpe}{\texttt{irc://freenode.net:ircd/\#gpe}}
}

\smallskip

Eine GUI-Umgebung, die auf Palmtops und Handheld-Computer wie den HP
iPAQ und den Sharp Zaurus abzielt. �GNOME f�r Handhelds� ist nicht
weit weg von der Wahrheit. GPE ist Freie Software.

\smallskip
\textbf{Eigenschaften}
\begin{compactitem}
   \item Eigene GTK-basierte Widget Library (LGPL)
   \item Applikationen: Kalendar, To-do-Liste, Adre�buch,
     Medienspieler, Notizbuch, Spiele und mehr (GPL)
   \item Systemwerkzeuge wie Wi-Fi- und Bluetooth-Verwaltung
   \item Internationalisierung
   \item �Display migration�: laufende Anwendungen auf andere
  Displays verschieben
\end{compactitem}

\smallskip
\textbf{Technik}
\begin{compactitem}
   \item X Window System{\texttrademark}{}-Server mit Render- und
     Resize/Rotate-Erweiterungen (z.\,B. freedesktop.org-X-Server, kdrive)
   \item fontconfig
   \item XFT2 mit Subpixel-anti-aliasing f�r Schriften
   \item GTK+~2.2, bidirektionaler Text, �display migration�
     usw.
   \item Unterst�tzung von freedesktop.org-Standards
   \item Matchbox, ein schlanker Fenstermanager, der f�r
     Embedded-Ger�te entworfen wurde
   \item SQLite zur Datenspeicherung
   \item GStreamer Multimedia-Framework
   \item Pakete f�r Familiar Linux verf�gbar;
     OpenEmbedded-Unterst�tzung in Arbeit.
   \item \textbf{NEU! Synchronisierung f�r Kontakte, To-Do und
     Kalender} in beide Richtungen mittels multisync.
\end{compactitem}

\smallskip
\textit{Anwender und Entwickler willkommen!}%
\hfill\tiny{\rcsInfoRevision, \rcsInfoYear-\rcsInfoMonth-\rcsInfoDay}
}

%%% Local Variables: 
%%% mode: latex
%%% TeX-master: "gpe-flyer"
%%% End: 

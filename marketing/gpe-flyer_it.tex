% Italian version by Giansalvo Gusinu <giansalvo at gusinu.net>
% translation of gpe-flyer.tex Revision: 1.10, Tue Apr 13 19:39:44 2004 UTC
\rcsInfo $Id$
\renewcommand{\flyertext}{%
%\selectlanguage{french}
%
%\AddToShipoutPicture{%
%\includegraphics[width=\textwidth/2-1cm]{gpe-logo-simple.jpg}}
\textbf{\large
  \includegraphics[height=\lineheight*3/5]{gpe-logo-simple.jpg}
GPE: Ambiente grafico per palmari
  \includegraphics[height=\lineheight*3/5]{gpe-logo-simple.jpg}
}

\smallskip
\small{
  \href{http://gpe.handhelds.org}{\texttt{http://gpe.handhelds.org}} --
  \href{mailto:gpe@handhelds.org}{\texttt{mailto:gpe@handhelds.org}} -- 
  \href{irc://freenode.net:ircd/\#gpe}{\texttt{irc://freenode.net:ircd/\#gpe}}
}

\smallskip
Ambiente grafico destinato a computer palmari tipo iPAQ di HP e Zaurus di Sharp.
Rilasciato con licenza 'free/open source'. Pensate a {\guillemotright}GNOME per palmari'{\guillemotleft}.

\smallskip
\textbf{Caratteristiche principali}
\begin{compactitem}
   \item Libreria specifica di widget grafici basata su GTK (LGPL)
   \item Applicazioni: calendario, agenda, gestione di task, rubrica indirizzi,
     lettore multimediale, bloc notes, giochi e altro (GPL)
   \item Utilit{\`a} di sistema tra cui WiFi manager et Bluetooth manager
   \item Internazionalizzazione
   \item Migrazione del display: trasferimento dell'output di un programma 
            in esecuzione su un altro schermo.
\end{compactitem}

\smallskip
\textbf{Tecnologia}
\begin{compactitem}
   \item Server X Window System{\texttrademark}{} (freedesktop.org X
     server, kdrive) con estensioni Render et Resize/Rotate
   \item fontconfig, XFT2 con supporto per font con subpixel anti-aliased
   \item GTK+~2.2, supporto per testo bidirezionale, migrazione del display, etc.
   \item supporto degli standard freedesktop.org
   \item Matchbox, un window manager leggero progettato per apparecchi embedded.
   \item SQLite per database
   \item framework multimediale GStreamer
   \item Pacchetti disponibili per la distribuzione Linux Familiar. Supporto per OpenEmbedded in preparazione.
   \item \textbf{NUOVO! Sincronizzazione dei Contatti, TODO e Calendario} 
   bidirezionale con multisync.
\end{compactitem}

\smallskip
\textit{{\textexclamdown}Utenti e sviluppatori sono benvenuti!}%
\hfill\tiny{Revision:~\rcsInfoRevision, \rcsInfoYear-\rcsInfoMonth-\rcsInfoDay}
}

%%% Local Variables: 
%%% mode: latex
%%% TeX-master: "gpe-flyer"
%%% End: 
